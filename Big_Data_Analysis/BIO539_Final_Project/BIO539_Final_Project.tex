\PassOptionsToPackage{unicode=true}{hyperref} % options for packages loaded elsewhere
\PassOptionsToPackage{hyphens}{url}
%
\documentclass[]{article}
\usepackage{lmodern}
\usepackage{amssymb,amsmath}
\usepackage{ifxetex,ifluatex}
\usepackage{fixltx2e} % provides \textsubscript
\ifnum 0\ifxetex 1\fi\ifluatex 1\fi=0 % if pdftex
  \usepackage[T1]{fontenc}
  \usepackage[utf8]{inputenc}
  \usepackage{textcomp} % provides euro and other symbols
\else % if luatex or xelatex
  \usepackage{unicode-math}
  \defaultfontfeatures{Ligatures=TeX,Scale=MatchLowercase}
\fi
% use upquote if available, for straight quotes in verbatim environments
\IfFileExists{upquote.sty}{\usepackage{upquote}}{}
% use microtype if available
\IfFileExists{microtype.sty}{%
\usepackage[]{microtype}
\UseMicrotypeSet[protrusion]{basicmath} % disable protrusion for tt fonts
}{}
\IfFileExists{parskip.sty}{%
\usepackage{parskip}
}{% else
\setlength{\parindent}{0pt}
\setlength{\parskip}{6pt plus 2pt minus 1pt}
}
\usepackage{hyperref}
\hypersetup{
            pdftitle={BIO\_539\_Final\_Project\_Piraino},
            pdfborder={0 0 0},
            breaklinks=true}
\urlstyle{same}  % don't use monospace font for urls
\usepackage[margin=1in]{geometry}
\usepackage{color}
\usepackage{fancyvrb}
\newcommand{\VerbBar}{|}
\newcommand{\VERB}{\Verb[commandchars=\\\{\}]}
\DefineVerbatimEnvironment{Highlighting}{Verbatim}{commandchars=\\\{\}}
% Add ',fontsize=\small' for more characters per line
\usepackage{framed}
\definecolor{shadecolor}{RGB}{248,248,248}
\newenvironment{Shaded}{\begin{snugshade}}{\end{snugshade}}
\newcommand{\AlertTok}[1]{\textcolor[rgb]{0.94,0.16,0.16}{#1}}
\newcommand{\AnnotationTok}[1]{\textcolor[rgb]{0.56,0.35,0.01}{\textbf{\textit{#1}}}}
\newcommand{\AttributeTok}[1]{\textcolor[rgb]{0.77,0.63,0.00}{#1}}
\newcommand{\BaseNTok}[1]{\textcolor[rgb]{0.00,0.00,0.81}{#1}}
\newcommand{\BuiltInTok}[1]{#1}
\newcommand{\CharTok}[1]{\textcolor[rgb]{0.31,0.60,0.02}{#1}}
\newcommand{\CommentTok}[1]{\textcolor[rgb]{0.56,0.35,0.01}{\textit{#1}}}
\newcommand{\CommentVarTok}[1]{\textcolor[rgb]{0.56,0.35,0.01}{\textbf{\textit{#1}}}}
\newcommand{\ConstantTok}[1]{\textcolor[rgb]{0.00,0.00,0.00}{#1}}
\newcommand{\ControlFlowTok}[1]{\textcolor[rgb]{0.13,0.29,0.53}{\textbf{#1}}}
\newcommand{\DataTypeTok}[1]{\textcolor[rgb]{0.13,0.29,0.53}{#1}}
\newcommand{\DecValTok}[1]{\textcolor[rgb]{0.00,0.00,0.81}{#1}}
\newcommand{\DocumentationTok}[1]{\textcolor[rgb]{0.56,0.35,0.01}{\textbf{\textit{#1}}}}
\newcommand{\ErrorTok}[1]{\textcolor[rgb]{0.64,0.00,0.00}{\textbf{#1}}}
\newcommand{\ExtensionTok}[1]{#1}
\newcommand{\FloatTok}[1]{\textcolor[rgb]{0.00,0.00,0.81}{#1}}
\newcommand{\FunctionTok}[1]{\textcolor[rgb]{0.00,0.00,0.00}{#1}}
\newcommand{\ImportTok}[1]{#1}
\newcommand{\InformationTok}[1]{\textcolor[rgb]{0.56,0.35,0.01}{\textbf{\textit{#1}}}}
\newcommand{\KeywordTok}[1]{\textcolor[rgb]{0.13,0.29,0.53}{\textbf{#1}}}
\newcommand{\NormalTok}[1]{#1}
\newcommand{\OperatorTok}[1]{\textcolor[rgb]{0.81,0.36,0.00}{\textbf{#1}}}
\newcommand{\OtherTok}[1]{\textcolor[rgb]{0.56,0.35,0.01}{#1}}
\newcommand{\PreprocessorTok}[1]{\textcolor[rgb]{0.56,0.35,0.01}{\textit{#1}}}
\newcommand{\RegionMarkerTok}[1]{#1}
\newcommand{\SpecialCharTok}[1]{\textcolor[rgb]{0.00,0.00,0.00}{#1}}
\newcommand{\SpecialStringTok}[1]{\textcolor[rgb]{0.31,0.60,0.02}{#1}}
\newcommand{\StringTok}[1]{\textcolor[rgb]{0.31,0.60,0.02}{#1}}
\newcommand{\VariableTok}[1]{\textcolor[rgb]{0.00,0.00,0.00}{#1}}
\newcommand{\VerbatimStringTok}[1]{\textcolor[rgb]{0.31,0.60,0.02}{#1}}
\newcommand{\WarningTok}[1]{\textcolor[rgb]{0.56,0.35,0.01}{\textbf{\textit{#1}}}}
\usepackage{graphicx,grffile}
\makeatletter
\def\maxwidth{\ifdim\Gin@nat@width>\linewidth\linewidth\else\Gin@nat@width\fi}
\def\maxheight{\ifdim\Gin@nat@height>\textheight\textheight\else\Gin@nat@height\fi}
\makeatother
% Scale images if necessary, so that they will not overflow the page
% margins by default, and it is still possible to overwrite the defaults
% using explicit options in \includegraphics[width, height, ...]{}
\setkeys{Gin}{width=\maxwidth,height=\maxheight,keepaspectratio}
\setlength{\emergencystretch}{3em}  % prevent overfull lines
\providecommand{\tightlist}{%
  \setlength{\itemsep}{0pt}\setlength{\parskip}{0pt}}
\setcounter{secnumdepth}{0}
% Redefines (sub)paragraphs to behave more like sections
\ifx\paragraph\undefined\else
\let\oldparagraph\paragraph
\renewcommand{\paragraph}[1]{\oldparagraph{#1}\mbox{}}
\fi
\ifx\subparagraph\undefined\else
\let\oldsubparagraph\subparagraph
\renewcommand{\subparagraph}[1]{\oldsubparagraph{#1}\mbox{}}
\fi

% set default figure placement to htbp
\makeatletter
\def\fps@figure{htbp}
\makeatother


\title{BIO\_539\_Final\_Project\_Piraino}
\author{}
\date{\vspace{-2.5em}}

\begin{document}
\maketitle

\hypertarget{load-in-all-the-needed-packages-in-order-to-run-the-upcoming-scripts.}{%
\subsection{Load in all the needed packages in order to run the upcoming
scripts.}\label{load-in-all-the-needed-packages-in-order-to-run-the-upcoming-scripts.}}

\begin{Shaded}
\begin{Highlighting}[]
\CommentTok{#install.packages("ggplot2")}
\KeywordTok{library}\NormalTok{(ggplot2)}
\CommentTok{#install.packages("tidyverse")}
\KeywordTok{library}\NormalTok{(tidyverse)}
\end{Highlighting}
\end{Shaded}

\begin{verbatim}
## -- Attaching packages ---------------------------------------------------------------------------- tidyverse 1.3.0 --
\end{verbatim}

\begin{verbatim}
## v tibble  3.0.1     v dplyr   0.8.5
## v tidyr   1.0.2     v stringr 1.4.0
## v readr   1.3.1     v forcats 0.5.0
## v purrr   0.3.4
\end{verbatim}

\begin{verbatim}
## -- Conflicts ------------------------------------------------------------------------------- tidyverse_conflicts() --
## x dplyr::filter() masks stats::filter()
## x dplyr::lag()    masks stats::lag()
\end{verbatim}

\begin{Shaded}
\begin{Highlighting}[]
\CommentTok{#install.packages("latexpdf")}
\KeywordTok{library}\NormalTok{(latexpdf)}
\NormalTok{Binding_Data_For_R_Project_}\DecValTok{4}\NormalTok{ <-}\StringTok{ }\KeywordTok{read_csv}\NormalTok{(}\StringTok{"Big_Data_Analysis/BIO539_Final_Project/Binding_Data_For_R_Project_4.csv"}\NormalTok{)}
\end{Highlighting}
\end{Shaded}

\begin{verbatim}
## Parsed with column specification:
## cols(
##   ClpX_Mutant = col_character(),
##   MqsA_1_76_GFP_Retained = col_double(),
##   SD = col_double()
## )
\end{verbatim}

\begin{Shaded}
\begin{Highlighting}[]
\NormalTok{ATP_Hydrolysis_Data_for_R_Project_}\DecValTok{3}\NormalTok{ <-}\StringTok{ }\KeywordTok{read_csv}\NormalTok{(}\StringTok{"Big_Data_Analysis/BIO539_Final_Project/ATP Hydrolysis Data for R Project 3.csv"}\NormalTok{)}
\end{Highlighting}
\end{Shaded}

\begin{verbatim}
## Parsed with column specification:
## cols(
##   ClpX_Mutant = col_character(),
##   Avg_Hydrolysis = col_double(),
##   SD = col_double()
## )
\end{verbatim}

\begin{Shaded}
\begin{Highlighting}[]
\NormalTok{MqsA_NTD_Degradation_R_Project_}\DecValTok{2}\NormalTok{ <-}\StringTok{ }\KeywordTok{read_csv}\NormalTok{(}\StringTok{"Big_Data_Analysis/BIO539_Final_Project/MqsA_NTD_Degradation_R_Project_2.csv"}\NormalTok{)}
\end{Highlighting}
\end{Shaded}

\begin{verbatim}
## Parsed with column specification:
## cols(
##   ClpX_Mutant = col_character(),
##   MqsA_NTD_Degradation = col_double(),
##   SD = col_double()
## )
\end{verbatim}

\hypertarget{plot-the-first-data-set-in-this-case-the-fluorescence-binding-assay-data}{%
\subsection{Plot the first data set, in this case, the fluorescence
binding assay
data}\label{plot-the-first-data-set-in-this-case-the-fluorescence-binding-assay-data}}

\begin{Shaded}
\begin{Highlighting}[]
\KeywordTok{ggplot}\NormalTok{(Binding_Data_For_R_Project_}\DecValTok{4}\NormalTok{) }\OperatorTok{+}\StringTok{ }
\StringTok{  }\KeywordTok{geom_bar}\NormalTok{( }\KeywordTok{aes}\NormalTok{(ClpX_Mutant,MqsA_}\DecValTok{1}\NormalTok{_}\DecValTok{76}\NormalTok{_GFP_Retained), }\DataTypeTok{stat=}\StringTok{"identity"}\NormalTok{, }\DataTypeTok{fill=}\StringTok{"skyblue"}\NormalTok{, }\DataTypeTok{alpha=}\FloatTok{0.7}\NormalTok{) }\OperatorTok{+}
\StringTok{  }\KeywordTok{geom_errorbar}\NormalTok{( }\KeywordTok{aes}\NormalTok{(}\DataTypeTok{x=}\NormalTok{ClpX_Mutant, }\DataTypeTok{ymin=}\NormalTok{MqsA_}\DecValTok{1}\NormalTok{_}\DecValTok{76}\NormalTok{_GFP_Retained}\OperatorTok{-}\NormalTok{SD, }\DataTypeTok{ymax=}\NormalTok{MqsA_}\DecValTok{1}\NormalTok{_}\DecValTok{76}\NormalTok{_GFP_Retained}\OperatorTok{+}\NormalTok{SD),}
                 \DataTypeTok{width=}\FloatTok{0.4}\NormalTok{, }\DataTypeTok{color=}\StringTok{"orange"}\NormalTok{, }\DataTypeTok{alpha=}\FloatTok{0.9}\NormalTok{, }\DataTypeTok{size=}\FloatTok{1.3}\NormalTok{) }\OperatorTok{+}
\StringTok{  }\KeywordTok{labs}\NormalTok{(}\DataTypeTok{title =} \StringTok{"Binding and Retention of GFP-MqsA 1-76 by ClpX N-terminal Mutants"}\NormalTok{ , }
       \DataTypeTok{x =} \StringTok{"ClpX Mutant"}\NormalTok{ , }\DataTypeTok{y =} \StringTok{"GFP-MqsA 1-76 Retained (A.U.)"}\NormalTok{)}
\end{Highlighting}
\end{Shaded}

\includegraphics{BIO539_Final_Project_files/figure-latex/unnamed-chunk-2-1.pdf}
\#This plot shows the various ClpX mutants on the x-axis and the amount
of fluorescence on the y-axis which is a direct representation of how
much substrate was retained by the specific mutant protein. \#This is
the standard format for the remaining data sets as well. Each will be
plotted with the ClpX mutants on the x-axis and the specific assay units
on the y-axis. The other bit of code within the plot script determines
the colors of the bars because this plot is a bar graph, and the width
of the bars and intensity of the color. The plot also contains error
bars which are determined by the y-min and y-max, which is the specific
y-value for the mutant plus and minus the standard deviation for that
mutant to determine the height above and below the y-value for the error
bar to represent.

\#You need to input the file name and the x- and y-axis names in the
specified spots in order for this plot to work correctly.

\#From the data you can see that the A30S mutation has the highest
retention rate. Although this mutant has a high error bar, there are
only two replicates run for this mutant while the others have 5
replicates and have dropped the high and the low. Even with the high
error bar, the low end of the A30S error bar is still higher than all
the other samples, showing that a polar mutation at the 30th residue
plays an important role in the binding of GFP-MqsA 1-76. The other
mutants all float moderately close to WT with no conclusions to be made.

\hypertarget{generating-a-summary-of-the-fluorescence-retained-for-the-first-assay}{%
\subsection{Generating a summary of the fluorescence retained for the
first
assay}\label{generating-a-summary-of-the-fluorescence-retained-for-the-first-assay}}

\begin{Shaded}
\begin{Highlighting}[]
\NormalTok{Binding_Data_For_R_Project_}\DecValTok{4} \OperatorTok
\StringTok{  }\KeywordTok{summary}\NormalTok{(MqsA_}\DecValTok{1}\NormalTok{_}\DecValTok{76}\NormalTok{_GFP_Retained)}
\end{Highlighting}
\end{Shaded}

\begin{verbatim}
##  ClpX_Mutant        MqsA_1_76_GFP_Retained       SD       
##  Length:6           Min.   : 2.612         Min.   : 1.07  
##  Class :character   1st Qu.:13.733         1st Qu.: 5.13  
##  Mode  :character   Median :18.017         Median : 6.93  
##                     Mean   :26.175         Mean   :12.65  
##                     3rd Qu.:22.163         3rd Qu.:12.88  
##                     Max.   :82.551         Max.   :41.61
\end{verbatim}

\#This is not a necessary step but can be useful to visualize the mean
fluorescence units retained across all mutants, the more data that is
amassed.

\#This process of plotting the data set and then calculating the summary
will be repeated for the other data sets prior to combining data sets.

\hypertarget{plot-the-atp-hydrolysis-data-for-the-clpx-mutants}{%
\subsection{Plot the ATP hydrolysis data for the ClpX
mutants}\label{plot-the-atp-hydrolysis-data-for-the-clpx-mutants}}

\begin{Shaded}
\begin{Highlighting}[]
\KeywordTok{ggplot}\NormalTok{(ATP_Hydrolysis_Data_for_R_Project_}\DecValTok{3}\NormalTok{) }\OperatorTok{+}\StringTok{ }
\StringTok{  }\KeywordTok{geom_bar}\NormalTok{( }\KeywordTok{aes}\NormalTok{(ClpX_Mutant,Avg_Hydrolysis), }\DataTypeTok{stat=}\StringTok{"identity"}\NormalTok{, }\DataTypeTok{fill=}\StringTok{"skyblue"}\NormalTok{, }\DataTypeTok{alpha=}\FloatTok{0.7}\NormalTok{) }\OperatorTok{+}
\StringTok{  }\KeywordTok{geom_errorbar}\NormalTok{( }\KeywordTok{aes}\NormalTok{(}\DataTypeTok{x=}\NormalTok{ClpX_Mutant, }\DataTypeTok{ymin=}\NormalTok{Avg_Hydrolysis}\OperatorTok{-}\NormalTok{SD, }\DataTypeTok{ymax=}\NormalTok{Avg_Hydrolysis}\OperatorTok{+}\NormalTok{SD),}
                 \DataTypeTok{width=}\FloatTok{0.4}\NormalTok{, }\DataTypeTok{color=}\StringTok{"orange"}\NormalTok{, }\DataTypeTok{alpha=}\FloatTok{0.9}\NormalTok{, }\DataTypeTok{size=}\FloatTok{1.3}\NormalTok{) }\OperatorTok{+}
\StringTok{  }\KeywordTok{labs}\NormalTok{(}\DataTypeTok{title =} \StringTok{"ATP Hydrolysis Rate at Ten Minutes by ClpX N-terminal Mutants"}\NormalTok{ , }
       \DataTypeTok{x =} \StringTok{"ClpX Mutant"}\NormalTok{ , }\DataTypeTok{y =} \StringTok{"nmol Pi"}\NormalTok{)}
\end{Highlighting}
\end{Shaded}

\begin{verbatim}
## Warning: Removed 1 rows containing missing values (position_stack).
\end{verbatim}

\includegraphics{BIO539_Final_Project_files/figure-latex/unnamed-chunk-4-1.pdf}
\#This plots the ClpX mutants on the x-axis and the average ATP
hydrolysis in nmol Pi on the y-axis. Refer to the plotting of the
fluorescence data to see what the other pieces of code here represent.

\#The results of this plot show that the L13A mutant has a much higher
ATP hydrolysis rate over the 10 minute hydrolysis reaction than the
other mutations. H23A is also increased compared to WT Clpx. Here, A30S
has no rate because it has not yet been tested for ATP hydrolysis. The
data will be updated when that assay has been run. The hydrolysis of ATP
is vital to the role of ClpX in unfolding substrates for degradation by
ClpP but has not been shown to have any effect on actually recognizing
and binding substrates which we will look at further on.

\hypertarget{calculate-the-summary-of-the-average-hydrolysis-for-all-clpx-mutants}{%
\subsection{Calculate the summary of the average hydrolysis for all ClpX
mutants}\label{calculate-the-summary-of-the-average-hydrolysis-for-all-clpx-mutants}}

\begin{Shaded}
\begin{Highlighting}[]
\NormalTok{ATP_Hydrolysis_Data_for_R_Project_}\DecValTok{3} \OperatorTok
\StringTok{  }\KeywordTok{summary}\NormalTok{(Avg_Hydrolysis)}
\end{Highlighting}
\end{Shaded}

\begin{verbatim}
##  ClpX_Mutant        Avg_Hydrolysis         SD         
##  Length:6           Min.   :0.0527   Min.   :0.06083  
##  Class :character   1st Qu.:0.4467   1st Qu.:0.06627  
##  Mode  :character   Median :0.6500   Median :0.09538  
##                     Mean   :0.6542   Mean   :0.09339  
##                     3rd Qu.:0.9217   3rd Qu.:0.10764  
##                     Max.   :1.2000   Max.   :0.13682  
##                     NA's   :1        NA's   :1
\end{verbatim}

\hypertarget{plot-the-degradation-assays-for-the-clpx-mutants-using-mqsa-ntd-as-the-substrate}{%
\subsection{Plot the degradation assays for the ClpX mutants using
MqsA-NTD as the
substrate}\label{plot-the-degradation-assays-for-the-clpx-mutants-using-mqsa-ntd-as-the-substrate}}

\begin{Shaded}
\begin{Highlighting}[]
\KeywordTok{ggplot}\NormalTok{(MqsA_NTD_Degradation_R_Project_}\DecValTok{2}\NormalTok{) }\OperatorTok{+}\StringTok{ }
\StringTok{  }\KeywordTok{geom_bar}\NormalTok{( }\KeywordTok{aes}\NormalTok{(ClpX_Mutant,MqsA_NTD_Degradation), }\DataTypeTok{stat=}\StringTok{"identity"}\NormalTok{, }\DataTypeTok{fill=}\StringTok{"skyblue"}\NormalTok{, }\DataTypeTok{alpha=}\FloatTok{0.7}\NormalTok{) }\OperatorTok{+}
\StringTok{  }\KeywordTok{geom_errorbar}\NormalTok{( }\KeywordTok{aes}\NormalTok{(}\DataTypeTok{x=}\NormalTok{ClpX_Mutant, }\DataTypeTok{ymin=}\NormalTok{MqsA_NTD_Degradation}\OperatorTok{-}\NormalTok{SD, }\DataTypeTok{ymax=}\NormalTok{MqsA_NTD_Degradation}\OperatorTok{+}\NormalTok{SD),}
                 \DataTypeTok{width=}\FloatTok{0.4}\NormalTok{, }\DataTypeTok{color=}\StringTok{"orange"}\NormalTok{, }\DataTypeTok{alpha=}\FloatTok{0.9}\NormalTok{, }\DataTypeTok{size=}\FloatTok{1.3}\NormalTok{) }\OperatorTok{+}
\StringTok{  }\KeywordTok{labs}\NormalTok{(}\DataTypeTok{title =} \StringTok{"MqsA-NTD Degradation After 180 min. Incubation with ClpX Mutants"}\NormalTok{ , }
       \DataTypeTok{x =} \StringTok{"ClpX Mutant"}\NormalTok{ , }\DataTypeTok{y =} \StringTok{"Percent Substrate Remaining"}\NormalTok{)}
\end{Highlighting}
\end{Shaded}

\begin{verbatim}
## Warning: Removed 2 rows containing missing values (position_stack).
\end{verbatim}

\includegraphics{BIO539_Final_Project_files/figure-latex/unnamed-chunk-6-1.pdf}
\#This plots the ClpX mutants on the x-axis and the percent of the
MqsA-NTD substrate remaining after 180 minutes on the y-axis. Refer to
the plotting of the fluorescence data to see what the other pieces of
code here represent.

\#The results of this plot show that H23A has highest degradation rate
out of all the ClpX mutants. This is interesting because H23A also had a
high ATP hydrolysis rate and ATP hydrolysis is directly needed in order
for ClpX to unfold its substrates for degradation by ClpP so the
beginnings of a correlation are shown here. Unfortunately, A30S and L13A
have not been tested in this assay yet but the data set and plot will be
updated when those assays are run.

\hypertarget{calculate-the-summary-of-the-percent-of-substrate-remaining-for-all-clpx-mutants}{%
\subsection{Calculate the summary of the percent of substrate remaining
for all ClpX
mutants}\label{calculate-the-summary-of-the-percent-of-substrate-remaining-for-all-clpx-mutants}}

\begin{Shaded}
\begin{Highlighting}[]
\NormalTok{MqsA_NTD_Degradation_R_Project_}\DecValTok{2} \OperatorTok
\StringTok{  }\KeywordTok{summary}\NormalTok{(MqsA_NTD_Degradation)}
\end{Highlighting}
\end{Shaded}

\begin{verbatim}
##  ClpX_Mutant        MqsA_NTD_Degradation       SD         
##  Length:6           Min.   :0.5023       Min.   :0.00000  
##  Class :character   1st Qu.:0.6194       1st Qu.:0.03456  
##  Mode  :character   Median :0.6680       Median :0.07377  
##                     Mean   :0.6574       Mean   :0.06952  
##                     3rd Qu.:0.7060       3rd Qu.:0.10872  
##                     Max.   :0.7912       Max.   :0.13053  
##                     NA's   :2            NA's   :2
\end{verbatim}

\hypertarget{combine-the-fluorescence-data-set-and-the-atp-hydrolysis-data-set}{%
\subsection{Combine the fluorescence data set and the ATP hydrolysis
data
set}\label{combine-the-fluorescence-data-set-and-the-atp-hydrolysis-data-set}}

\begin{Shaded}
\begin{Highlighting}[]
\NormalTok{ATP_vs_Binding <-}\StringTok{ }\KeywordTok{left_join}\NormalTok{(ATP_Hydrolysis_Data_for_R_Project_}\DecValTok{3}\NormalTok{, Binding_Data_For_R_Project_}\DecValTok{4}\NormalTok{,}
                            \DataTypeTok{by =} \KeywordTok{c}\NormalTok{(}\StringTok{"ClpX_Mutant"}\NormalTok{,}\StringTok{"ClpX_Mutant"}\NormalTok{))}
\end{Highlighting}
\end{Shaded}

\#This has joined the two data sets by the common denominator column,
``ClpX\_Mutant''

\#This provides a new dataset with 5 columns, one for the ClpX\_Mutant
joining column, then one for each of the recorded data for the
respective assays and one for each of their respective standard
deviations

\hypertarget{plot-the-combined-atp-binding-data-set}{%
\subsection{Plot the combined ATP-Binding data
set}\label{plot-the-combined-atp-binding-data-set}}

\begin{Shaded}
\begin{Highlighting}[]
\KeywordTok{ggplot}\NormalTok{(ATP_vs_Binding) }\OperatorTok{+}\StringTok{ }
\StringTok{  }\KeywordTok{geom_point}\NormalTok{( }\KeywordTok{aes}\NormalTok{(MqsA_}\DecValTok{1}\NormalTok{_}\DecValTok{76}\NormalTok{_GFP_Retained,Avg_Hydrolysis, }\DataTypeTok{color =}\NormalTok{ ClpX_Mutant)) }\OperatorTok{+}
\StringTok{  }\KeywordTok{theme_bw}\NormalTok{() }\OperatorTok{+}\StringTok{ }
\StringTok{  }\KeywordTok{labs}\NormalTok{(}\DataTypeTok{title =} \StringTok{"How ClpX N-terminal Mutant ATP Hydrolysis Effects Binding of MqsA 1-76"}\NormalTok{ , }
       \DataTypeTok{x =} \StringTok{"GFP-MqsA 1-76 Retained (A.U.)"}\NormalTok{ , }\DataTypeTok{y =} \StringTok{"ATP Hydrolysis (nmol Pi)"}\NormalTok{, }\DataTypeTok{color =} \StringTok{" "}\NormalTok{) }\OperatorTok{+}
\StringTok{  }\KeywordTok{theme}\NormalTok{(}\DataTypeTok{legend.position =} \StringTok{"right"}\NormalTok{)}
\end{Highlighting}
\end{Shaded}

\begin{verbatim}
## Warning: Removed 1 rows containing missing values (geom_point).
\end{verbatim}

\includegraphics{BIO539_Final_Project_files/figure-latex/unnamed-chunk-9-1.pdf}
\#This script plots the newly combined data set by putting the data from
the fluorescence assay on the x-axis and the data from the ATP
hydrolysis assay on the y-axis with the ClpX\_Mutant column denoting the
color and therefore allowing determination of which sample is which.

\#This plot is not overly informative because of the fact that ATP
hydrolysis and binding of substrates does not have any correlation,
however, it is worth observing to see any possible trends of increased
or decreased activity in both areas. From this plot, it does not look
like there is any correlation to the amount of ATP hydrolyzed and the
amount of GFP-MqsA 1-76 bound by any of the ClpX mutants.

\hypertarget{combine-the-fluorescence-data-set-and-the-degradation-data-set}{%
\subsection{Combine the fluorescence data set and the degradation data
set}\label{combine-the-fluorescence-data-set-and-the-degradation-data-set}}

\begin{Shaded}
\begin{Highlighting}[]
\NormalTok{Degradation_vs_Binding <-}\StringTok{ }\KeywordTok{left_join}\NormalTok{(MqsA_NTD_Degradation_R_Project_}\DecValTok{2}\NormalTok{, Binding_Data_For_R_Project_}\DecValTok{4}\NormalTok{,}
                            \DataTypeTok{by =} \KeywordTok{c}\NormalTok{(}\StringTok{"ClpX_Mutant"}\NormalTok{,}\StringTok{"ClpX_Mutant"}\NormalTok{))}
\end{Highlighting}
\end{Shaded}

\#This has joined the two data sets by the common denominator column,
``ClpX\_Mutant''

\#This provides a new dataset with 5 columns, one for the ClpX\_Mutant
joining column, then one for each of the recorded data for the
respective assays and one for each of their respective standard
deviations

\hypertarget{plot-the-combined-degradation-binding-data-set}{%
\subsection{Plot the combined Degradation-binding data
set}\label{plot-the-combined-degradation-binding-data-set}}

\begin{Shaded}
\begin{Highlighting}[]
\KeywordTok{ggplot}\NormalTok{(Degradation_vs_Binding) }\OperatorTok{+}\StringTok{ }
\StringTok{  }\KeywordTok{geom_point}\NormalTok{( }\KeywordTok{aes}\NormalTok{(MqsA_}\DecValTok{1}\NormalTok{_}\DecValTok{76}\NormalTok{_GFP_Retained,MqsA_NTD_Degradation, }\DataTypeTok{color =}\NormalTok{ ClpX_Mutant)) }\OperatorTok{+}
\StringTok{  }\KeywordTok{theme_bw}\NormalTok{() }\OperatorTok{+}\StringTok{ }
\StringTok{  }\KeywordTok{labs}\NormalTok{(}\DataTypeTok{title =} \StringTok{"How the binding of GFP-MqsA 1-76 by ClpX N-terminal mutants relates to MqsA-NTD degradation"}\NormalTok{ , }
       \DataTypeTok{x =} \StringTok{"GFP-MqsA 1-76 Retained (A.U.)"}\NormalTok{ , }\DataTypeTok{y =} \StringTok{"Percent Substrate Remaining"}\NormalTok{, }\DataTypeTok{color =} \StringTok{" "}\NormalTok{) }\OperatorTok{+}
\StringTok{  }\KeywordTok{theme}\NormalTok{(}\DataTypeTok{legend.position =} \StringTok{"right"}\NormalTok{)}
\end{Highlighting}
\end{Shaded}

\begin{verbatim}
## Warning: Removed 2 rows containing missing values (geom_point).
\end{verbatim}

\includegraphics{BIO539_Final_Project_files/figure-latex/unnamed-chunk-11-1.pdf}
\#This script plots the newly combined data set by putting the data from
the fluorescence assay on the x-axis and the data from the MqsA-NTD
degradation assay on the y-axis with the ClpX\_Mutant column denoting
the color and therefore allowing determination of which sample is which.

\#This plot is particularly interesting because the amount of GFP-MqsA
1-76 retained by the ClpX variant should directly correlate with the
amount of MqsA-NTD that is degraded, assuming that the specific mutation
in the ClpX protein does not effect the unfolding and transfer of a
bound substrate. From the plot, it appears that the WT ClpX has the best
degradation of MqsA-NTD while still achieving a high binding affinity
for GFP-MqsA 1-76. The tested mutations appear to have some defect in
the steps going from binding the substrate to unfolding and transfering
it to ClpP for degradation.

\#It is worth noting that although the mutations, A30S and L13A were
tested in the fluorescence assay, they were not tested in the
degradation assay and therefore do not appear on this plot. Once they
are tested in that assay, the dataset and plot will be updated

\hypertarget{combine-degradation-dataset-and-the-atp-hydrolysis-data-set}{%
\subsection{Combine degradation dataset and the ATP hydrolysis data
set}\label{combine-degradation-dataset-and-the-atp-hydrolysis-data-set}}

\begin{Shaded}
\begin{Highlighting}[]
\NormalTok{Degradation_vs_Hydrolysis <-}\StringTok{ }\KeywordTok{left_join}\NormalTok{(MqsA_NTD_Degradation_R_Project_}\DecValTok{2}\NormalTok{, ATP_Hydrolysis_Data_for_R_Project_}\DecValTok{3}\NormalTok{,}
                                       \DataTypeTok{by =} \KeywordTok{c}\NormalTok{(}\StringTok{"ClpX_Mutant"}\NormalTok{, }\StringTok{"ClpX_Mutant"}\NormalTok{))}
\end{Highlighting}
\end{Shaded}

\#This has joined the two data sets by the common denominator column,
``ClpX\_Mutant''

\#This provides a new dataset with 5 columns, one for the ClpX\_Mutant
joining column, then one for each of the recorded data for the
respective assays and one for each of their respective standard
deviations

\hypertarget{plot-the-combined-degradation-atp-hydrolysis-data-set}{%
\subsection{Plot the combined degradation-ATP hydrolysis data
set}\label{plot-the-combined-degradation-atp-hydrolysis-data-set}}

\begin{Shaded}
\begin{Highlighting}[]
\KeywordTok{ggplot}\NormalTok{(Degradation_vs_Hydrolysis) }\OperatorTok{+}\StringTok{ }
\StringTok{  }\KeywordTok{geom_point}\NormalTok{( }\KeywordTok{aes}\NormalTok{(Avg_Hydrolysis,MqsA_NTD_Degradation, }\DataTypeTok{color =}\NormalTok{ ClpX_Mutant)) }\OperatorTok{+}
\StringTok{  }\KeywordTok{theme_bw}\NormalTok{() }\OperatorTok{+}\StringTok{ }
\StringTok{  }\KeywordTok{labs}\NormalTok{(}\DataTypeTok{title =} \StringTok{"How ClpX N-terminal Mutant ATP Hydrolysis Effects Degradation of MqsA-NTD"}\NormalTok{ , }
       \DataTypeTok{x =} \StringTok{"ATP Hydrolysis (nmol Pi)"}\NormalTok{ , }\DataTypeTok{y =} \StringTok{"Percent Substrate Remaining"}\NormalTok{, }\DataTypeTok{color =} \StringTok{" "}\NormalTok{) }\OperatorTok{+}
\StringTok{  }\KeywordTok{theme}\NormalTok{(}\DataTypeTok{legend.position =} \StringTok{"right"}\NormalTok{)}
\end{Highlighting}
\end{Shaded}

\begin{verbatim}
## Warning: Removed 2 rows containing missing values (geom_point).
\end{verbatim}

\includegraphics{BIO539_Final_Project_files/figure-latex/unnamed-chunk-13-1.pdf}
\#This script plots the newly combined data set by putting the data from
the ATP hydrolysis assay on the x-axis and the data from the MqsA-NTD
degradation assay on the y-axis with the ClpX\_Mutant column denoting
the color and therefore allowing determination of which sample is which.

\#The results of this plot show that H23A has the largest correlation
between ATP hydrolysis rate and MqsA-NTD degradation. It has the highest
hydrolysis rate and the most degradation of the substrate. ClpX utilizes
ATP to power the unfolding of substrates to then be transfered to ClpP
for degradation so this correlation makes sense in terms of proteolytic
mechanisms. Again, A30S and L13A were not tested in both assays so they
are ommitted from this plot. When those assays are run, the data set and
plot will be updated to reflect those results.

\end{document}
